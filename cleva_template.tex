\newcommand{\D}{11} % number of protocol dimensions (config option)
\newcommand{\U}{2} % number of scale units (config option)
%-$NUMBER-OF-METHODS$
\newcommand{\EV}{15} %number of evaluation measures (config option)

\newdimen\R % maximal diagram radius (config option)
\R=2.7cm
\newdimen\L % radius to put dimension labels (config option)
\L=3.3cm

\newcommand{\A}{360/\D} % calculated angle between dimension axes
\newcommand{\B}{360/\EV} % calculated angle between evaluation measure axes
\newcommand{\BM}{\B/\M}

\newcommand{\Doffset}{3*\A - 90}
\newcommand{\Instrip}{3.6cm}
\newcommand{\Outstrip}{3.85cm}
\newcommand{\nodefontsize}{|\tiny \selectfont|}

\usetikzlibrary{decorations.text, arrows.meta}

\tikzset{
  font={\tiny\selectfont},
  myarrow/.style={thick, -latex},
  whiteshell/.style={draw=white,fill=white,opacity=0.0},
  whitecircle/.style={draw=black,fill=white,circle, align=center,, inner sep=1pt, opacity=0.75},
  magentashell/.style={draw=magenta,fill=magenta,fill opacity=0.4,  opacity=0.3},
  greenshell/.style={draw=green!50!black,fill=green!50!black, fill opacity=0.4,  opacity=0.3},
  blueshell/.style={draw=blue!70!black, fill=blue!70!black, fill opacity=0.4,opacity=0.3},
  orangeshell/.style={draw=orange!90!black, fill=orange!80,fill opacity=0.4, opacity=0.3},
  cyanshell/.style={draw=cyan!90!black, fill=cyan!80!black,fill opacity=0.4, opacity=0.3},
  brownshell/.style={draw=brown!90!black, fill=brown!80!black,fill opacity=0.4, opacity=0.3},
  % #1=radius, #2=start angle, #3=end angle, #4=draw style,
  % #5 text colour, #6=text
  pics/strip/.style args = {#1,#2,#3,#4,#5,#6}{
       code = {
        \draw[#4] (#2:#1-1.25mm) arc (#2:#3:#1-1.25mm)
             -- (#3:#1) -- (#3:#1+1.25mm) arc (#3:#2:#1+1.25mm)
             -- (#2:#1) -- cycle;
        \path[
              decoration={text along path, text color=#5, text = {#6},
                          text align = {align = center}, raise = -0.3ex},
              decorate] (#2:#1) arc (#2:#3:#1);
       }
  }
}



\begin{tikzpicture}[scale=1]
  \path (0:0cm) coordinate (O); % define coordinate for origin

  % draw the spiderweb
  \foreach \X in {1,...,\D}{
    \draw [opacity=0.5](\X*\A:0) -- (\X*\A:\R);
  }

  \foreach \Y in {0,...,\U}{
    \foreach \X in {1,...,\D}{
      \path (\X*\A:\Y*\R/\U) coordinate (D\X-\Y);
      \fill (D\X-\Y) circle (1.5pt);
    }
    \draw [opacity=0.5] (0:\Y*\R/\U) \foreach \X in {1,...,\D}{
        -- (\X*\A:\Y*\R/\U)
    } -- cycle;
  }

  % define labels for each dimension axis (names config option)
  \path (1*\A:\L) node (L1)[yshift=-2ex,xshift=-2.25ex, rotate=\Doffset-2*\A] {Multiple Models};
  \path (2*\A:\L) node (L2)[yshift=-2.5ex, xshift=-1ex,rotate=\Doffset-\A] {Federated};
  \path (3*\A:\L) node (L3) [xshift=0ex, yshift=-2.75ex,  rotate=\Doffset]{Online};
  \path (4*\A:\L) node (L4) [xshift=1.5ex, yshift=-2.25ex, rotate=\Doffset+\A]{Open World};
  \path (5*\A:\L) node (L5) [xshift=2.5ex, yshift=-0.5ex, rotate=\Doffset+2*\A]{Multiple Modalities};
  \path (6*\A:\L) node (L6) [xshift=2.75ex, yshift=0.5ex, rotate=-2*\A-5]{Active Data Query};
  \path (7*\A:\L) node (L7) [xshift=2.75ex, yshift=1.25ex, rotate=5-\A]{Task Order Discovery};
  \path (8*\A:\L) node (L8) [xshift=2.0ex, yshift=2.5ex, rotate=0]{Task Agnostic};
  \path (9*\A:\L) node (L9) [xshift=0ex, yshift=2.75ex, rotate=-(\Doffset-1*\A)]{Episodic Memory};
  \path (10*\A:\L) node (L10) [xshift=-2.5ex,yshift=1ex, rotate=2*\A-15]{Generative};
  \path (11*\A:\L) node (L11)[yshift=-0.5ex,xshift=-2.75ex, rotate=\Doffset-3*\A] {Uncertainty};

  % for each sample case draw a path around the web along concrete values
  % for the individual dimensions. Each node along the path is labeled
  % with an identifier using the following scheme:
  %
  %   D<d>-<v>, dimension <d> a number between 1 and \D (#dimensions) and
  %             value <v> a number between 0 and \U (#scale units)
  %
  % The paths will be drawn half-opaque, so that overlapping parts will be
  % rendered in a composite color.

	% 1 - multiple models,  2 - federated,  3 - online , 4 - open world, 5 -multiple modalities, 6- active, 7 - task order, 8 - task agnostic,  9 - episodic memory,  10 - generative,  11 - uncertainty

    % Inner circle
    %-$INNER-CIRCLE$

    % Outer circle
    %-$OUTER-CIRCLE$


    %% OUTSIDE circle labels
    % Top half
    \pic at (0,0){strip={\Outstrip,  1*\B,  0*\B,whitecircle,black,\nodefontsize Parameters}};
    \pic at (0,0){strip={\Outstrip,  2*\B,  1*\B,whitecircle,black,\nodefontsize Compute time}};
    \pic at (0,0){strip={\Outstrip,  3*\B,  2*\B,whitecircle,black,\nodefontsize MAC operations}};
    \pic at (0,0){strip={\Outstrip,  4*\B,  3*\B,whitecircle,black,\nodefontsize Communication}};
    \pic at (0,0){strip={\Outstrip,  5*\B,  4*\B,whitecircle,black,\nodefontsize Forgetting}};
    \pic at (0,0){strip={\Outstrip,  6*\B,  5*\B,whitecircle,black,\nodefontsize Forward transfer}};
    \pic at (0,0){strip={\Outstrip,  7*\B,  6*\B,whitecircle,black,\nodefontsize Backward transfer}};
    \pic at (0,0){strip={\Outstrip,  8*\B,  7*\B,whitecircle,black,\nodefontsize Openness}};

    % Bottom half (invert direction here instead of (i * \B, i-1 * \B) do (i-1 * \B, i * \B)
    % for improved text label readability
    \pic at (0,0){strip={\Outstrip,  8*\B,  9*\B,whitecircle,black,\nodefontsize Data per task}};
    \pic at (0,0){strip={\Outstrip,  9*\B, 10*\B,whitecircle,black,\nodefontsize Task order}};
    \pic at (0,0){strip={\Outstrip, 10*\B, 11*\B,whitecircle,black,\nodefontsize Per task metrics}};
    \pic at (0,0){strip={\Outstrip, 11*\B, 12*\B,whitecircle,black,\nodefontsize Optimization steps}};
    \pic at (0,0){strip={\Outstrip, 12*\B, 13*\B,whitecircle,black,\nodefontsize Generated data}};
    \pic at (0,0){strip={\Outstrip, 13*\B, 14*\B,whitecircle,black,\nodefontsize Stored data}};
    \pic at (0,0){strip={\Outstrip, 14*\B, 15*\B,whitecircle,black,\nodefontsize Memory}};



    \newcommand{\lentry}[2]{%
      % #1: color, #2: label
      \scalebox{0.6}{\fcolorbox{#1!30}{#1!30}{\textcolor{#1!30}{\rule{\fontcharht\font`X}{\fontcharht\font`X}}}} #2
    }
    \coordinate (center) at (0,0);
    {\protect\NoHyper
    \node[] (legend) [below=4.05cm of center, align = left] {
      %-$LEGEND$
    };
    \protect\endNoHyper}


\end{tikzpicture}
